\documentclass[a4paper, 12pt]{ctexart}
\usepackage{geometry}
\usepackage{fontspec}
\usepackage{tabularx}
\usepackage{multirow}
\usepackage{longtable}
\usepackage{titlesec}
\usepackage{hyperref}
\usepackage{indentfirst}
\usepackage{multicol}
\usepackage{xcolor}

\renewcommand{\today}{\number \year 年 \ifcase \month \or 1月\or 2月\or 3月\or %
4月\or 5月 \or 6月\or 7月\or 8月\or 9月\or 10月\or 11月\or %
12月\fi} 

\setlength{\parindent}{0em}

\geometry{a4paper,left=0.5in,right=0.5in,top=1in,bottom=1in}

\newfontfamily\subsubsectionfont[Color=black]{Times New Roman}

\titleformat*{\subsubsection}{\bfseries\subsubsectionfont}

\setlength{\parindent}{0pt}

\hypersetup{
    colorlinks=true,
    linkcolor=blue,
    filecolor=blue,      
    urlcolor=blue,
    citecolor=cyan,
}

\begin{document}

% \songti
\fontspec{Times New Roman}

\thispagestyle{myheadings}
\markright{\today}

\centerline{\large\bf 韦子谦}
\centerline{\textbf{电话} 18679651087}
\centerline{\textbf{邮箱} \href{mailto: weiziqianpsych@outlook.com}{weiziqianpsych@outlook.com}}
\centerline{\textbf{地址} \ 北京市海淀区成府路20号 (邮编: 100083)}

\  \par 
\  \par 

\subsubsection*{ \ \ 教育经历}


\begin{tabularx}{\textwidth}{p{2cm} p{0.1cm} X}

    2019-2022 & \multicolumn{2}{X}{\textbf{华南师范大学}} \\
    & $\circ$ & 基础心理学 \ | \ 学术型硕士 \\
    & $\circ$ & \textbf{研究领域}: 学习心理,\textbf{导师}: 陈栩茜 \ 副教授  \\
    % & $\circ$ & \textbf{导师}: 陈栩茜 \ 副教授 \\
    % & $\circ$ & \textbf{Core Modules}: Frontiers of Psychological Research; Principles and Techniques of Cognitive Neuroscience; Social Research Methods; Statistical Theory and Data Analysis; Research Methods in Psychology. \\

    2015-2019 & \multicolumn{2}{X}{\textbf{江西师范大学}} \\
    & $\circ$ & 心理学 (师范) \ | \ 学士 \\
    % & $\circ$ & \textbf{Core Modules}: Introduction to Psychology; Experimental Psychology; Research Methods in Psychology
    
    \end{tabularx}



\  \par 
\  \par 

%%%%%%%%%%%%%%%%%%%%%%%%%%%%%%%%%%%%%%%%%%%%%%%%%%%%%%%%%%%%%%%

\subsubsection*{ \ \ 研究经历}

\begin{tabularx}{\textwidth}{p{2cm} p{0.1cm} X}

    2021-2022 & \multicolumn{2}{X}{\textbf{阅读前的丰富的外部线索促进记忆中的多文本整合:来自概念网络分析的证据 | 硕士论文,华南师范大学}} \\
    % & $\circ$ & MA thesis \\
    & $\circ$ & 通过三个在线实验测量不同水平的多文本整合 \\
    & $\circ$ & 通过一个自编的Python脚本将被试的写作内容转化为概念网络\\
    & $\circ$ & 结果发现,虽然外部线索促进了多文本整合的“量”,但整合的“质”是难以提升的 \\

    2020-2021 & \multicolumn{2}{X}{\textbf{思维倾向性对知识结构建构的影响 | 硕士研究者,华南师范大学}} \\
    & $\circ$ & 撰写研究计划书并获得了华南师范大学心理学院研究生科研创新项目的资助 \\
    & $\circ$ & 收集了约120名被试的数据,并采用Python和R进行数据处理和统计分析  \\
    & $\circ$ &  结果发现,不同思维倾向性的大学生都获得了较好的知识结构 \\
    
    2020 & \multicolumn{2}{X}{\textbf{外部表征对阅读历史文本时的概念结构的影响 | 硕士研究者,华南师范大学}} \\
    & $\circ$ & 负责撰写伦理申请书、数据收集以及统计分析(采用SPSS和JASP) \\
    & $\circ$ & 结果发现,相比大纲,概念图导致了更好的概念结构 \\

    2019 & \multicolumn{2}{X}{\textbf{不同视野的面部识别 | 硕士研究者,华南师范大学}} \\
    & $\circ$ & 硕士课程\textit{Research Method in Psychology}的课程论文,讲师:孟明 \ 教授 \\ 
    & $\circ$ & 作为小组的组长统筹整个研究 \\
    & $\circ$ & 采用Matlab处理面孔刺激图片,分为高、低空间频率的刺激 \\
    & $\circ$ & 设计学术海报、进行口头汇报和撰写三千字的课程论文 (皆用英文完成) \\
    

    2018-2019 & \multicolumn{2}{X}{\textbf{自我损耗对决策中的信息回避的影响 | 学士论文,江西师范大学}} \\
    % & $\circ$ & BA thesis \\
    % & $\circ$ & Generated the research idea and designed experiments independently by reviewing publications \\
    & $\circ$ & 采用E-prime编写实验程序并采用SPSS进行统计分析 \\
    & $\circ$ & 在论文答辩中获得了高分 \\

    \end{tabularx}

%%%%%%%%%%%%%%%%%%%%%%%%%%%%%%%%%%%%%%%%%%%%%%%%%%%%%%%%%%%%%%%
\subsubsection*{ \ \ 工作经历}

\begin{tabularx}{\textwidth}{p{2cm} p{0.1cm} X}

    2022 & \multicolumn{2}{X}{\textbf{科研助理,清华大学}} \\
    & $ \circ $ & 作为\href{https://www.ymhlab.com}{Youth Mental Health Lab}的一员,参与一个关于心理健康数字干预的项目 \\
    & $ \circ $ & 目前主要负责问卷、文稿和干预训练的材料准备和数据收集 \\
    
    % Research asistant of the \href{https://www.ymhlab.com}{Youth Mental Health Lab} \\

    2021 & \multicolumn{2}{X}{\textbf{助教,华南师范大学}} \\
    & $ \circ $ & 作为三门本科生课程——\textit{心理统计学、心理测量学的原理和应用、统计软件的应用}的助教,主要负责作业的批改以及为学生答疑,讲师:陈启山 \ 副教授 \\

    2018 & \multicolumn{2}{X}{\textbf{实习教师,南昌麻丘中学}} \\
    & $ \circ $ & 负责讲授初中生心理健康课程,以及心理健康报、心理活动周等工作

    \end{tabularx}

\  \par 

%%%%%%%%%%%%%%%%%%%%%%%%%%%%%%%%%%%%%%%%%%%%%%%%%%%%%%%%%%%%%%%

\subsubsection*{ \ \ 荣誉和获奖}

\begin{tabularx}{\textwidth}{p{2cm} X}

    2019-2021 & 学业奖学金 (二等和三等)\\

    2019 & 心理学院研究生科研创新项目 (一般项目) \\
    
    2018-2019 & 学业奖学金 (三等)
    
\end{tabularx}

\  \par

%%%%%%%%%%%%%%%%%%%%%%%%%%%%%%%%%%%%%%%%%%%%%%%%%%%%%%%%%%%%%%%

\subsubsection*{ \ \ 研究成果}

\begin{description}
    
    \item \textbf{Wei, Z.} Zhang, Y., Yin, S., Clariana, R. B., \& Chen, X. (submitted). The effect of reading prompt and post-reading task on multiple document integration: Evidence from concept network analysis. \textit{Educational Technology Research and Development}. [SSCI]

    \item Chen, X., Li, Z., \textbf{Wei, Z.}, \& Clariana, R. B*. (accepted). The influence of the conceptual structure of external representations when relearning history content. \textit{Educational Technology Research and Development}. [SSCI]

    \item \textbf{韦子谦}, 陈栩茜, \& Clariana, R., B. (2022).文本信息加工中的知识结构的测量. 心理科学, 45(02), 306-314. \href{http://www.psysci.org/CN/Y2022/V45/I2/306}{http://www.psysci.org/CN/Y2022/V45/I2/306}

    
    \end{description}

\  \par

%%%%%%%%%%%%%%%%%%%%%%%%%%%%%%%%%%%%%%%%%%%%%%%%%%%%%%%%%%%%%%%

\subsubsection*{ \ \ 技能}

\begin{tabularx}{\textwidth}{p{2cm} X}
    
    软件 & R (擅长), Python (擅长), MATLAB (一般), SPSS (一般), E-Prime (一般), Mplus (基础), Microsoft Words, Excel \& PowerPoint (擅长), LaTeX (一般) \\
    
    语言 &英语 (IELTS: 6.5), 中文 (母语) \\
    
    \end{tabularx}

% \  \par
\newpage

%%%%%%%%%%%%%%%%%%%%%%%%%%%%%%%%%%%%%%%%%%%%%%%%%%%%%%%%%%%%%%%

\subsubsection*{ \ \ 推荐人}

% \ \ Assoc. Prof. Xuqian Chen \par
% \ \ School of Psychology, South China Normal University \par
% \ \ Email: \href{mailto: chenxuqian@m.scnu.edu.cn}{chenxuqian@m.scnu.edu.cn} \par
% \ \ Research interest: language cognition and learning; concept structure in memory	\par

% \ \par

% \ \ Prof. Aitao Lu  \par
% \ \ School of Psychology, South China Normal University \par
% \ \ Email: \href{mailto: atlupsy@gmail.com}{atlupsy@gmail.com} \par
% \ \ Research interest: cognitive process of language; language cognition; teenager development \par

\begin{multicols}{2}
    
    \ \ \textbf{陈栩茜 \ 副教授} \par
    \ \ 华南师范大学,心理学院 \par
    \ \ 邮箱: \href{mailto: chenxuqian@m.scnu.edu.cn}{chenxuqian@m.scnu.edu.cn} \par
    \ \ 研究兴趣: 记忆中的概念结构	\par
    
    \columnbreak

    \ \ \textbf{陆爱桃 \ 教授}  \par
    \ \ 华南师范大学,心理学院 \par
    \ \ 邮箱: \href{mailto: atlupsy@gmail.com}{atlupsy@gmail.com} \par
    \ \ 研究兴趣: 语言的认知加工 \par
    
    \end{multicols}

\end{document}