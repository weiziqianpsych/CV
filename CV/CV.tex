\documentclass[a4paper, 12pt]{article}
\usepackage{geometry}
\usepackage{fontspec}
\usepackage{tabularx}
\usepackage{multirow}
\usepackage{longtable}
\usepackage{titlesec}
\usepackage{hyperref}
\usepackage{indentfirst}

\setlength{\parindent}{0em}

\geometry{a4paper,left=0.5in,right=0.5in,top=1in,bottom=1in}

\newfontfamily\subsubsectionfont[Color=black]{Times New Roman}
\titleformat*{\subsubsection}{\bfseries\subsubsectionfont}

\setlength{\parindent}{0pt}

\hypersetup{
    colorlinks=true,
    linkcolor=blue,
    filecolor=blue,      
    urlcolor=blue,
    citecolor=cyan,
}

\begin{document}

\fontspec{Times New Roman}

\centerline{\large\bf Ziqian Wei}
\centerline{\textbf{Tel} +86 18679651087}
\centerline{\textbf{Email} \href{mailto: weiziqianpsych@outlook.com}{weiziqianpsych@outlook.com}}
\centerline{\textbf{Address} Guangzhou City, China (Code: 510631)}

\  \par 
\  \par 

\subsubsection*{ \ \ EDUCATION}


\begin{tabularx}{\textwidth}{p{2cm} p{0.1cm} X}

    2019-2022 & \multicolumn{2}{X}{\textbf{South China Normal University, China}} \\
    & $\circ$ & MA in Psychology \\
    & $\circ$ & \textbf{Research filed}: reading and learning \\
    & $\circ$ & \textbf{Supervisor}: Assoc. Prof. Xuqian Chen \\
    % & $\circ$ & \textbf{Core Modules}: Frontiers of Psychological Research; Principles and Techniques of Cognitive Neuroscience; Social Research Methods; Statistical Theory and Data Analysis; Research Methods in Psychology. \\

    2015-2019 & \multicolumn{2}{X}{\textbf{Jiangxi Normal University, China}} \\
    & $\circ$ & BA in Psychology \\
    % & $\circ$ & \textbf{Core Modules}: Introduction to Psychology; Experimental Psychology; Research Methods in Psychology
    
    \end{tabularx}



\  \par 
\  \par 

%%%%%%%%%%%%%%%%%%%%%%%%%%%%%%%%%%%%%%%%%%%%%%%%%%%%%%%%%%%%%%%

\subsubsection*{ \ \ RESEARCH EXPERIENCE}

\begin{tabularx}{\textwidth}{p{2cm} p{0.1cm} X}

    2021-2022 & \multicolumn{2}{X}{\textbf{Effects of reading purposes on integration from multiple documents: Evidence from concept network analysis}} \\
    & $\circ$ & MA thesis \\
    & $\circ$ & Conducted three online experiments to measure different aspects of integration \\
    & $\circ$ & Converted participants' summaries to networks via a self-made Python package\\
    & $\circ$ & Found that whereas reading purpose conditions influence the \textit{inclination} of integration, the \textit{quality} of integration was difficult to improve  \\

    2020-2021 & \multicolumn{2}{X}{\textbf{Thinking patterns influence knowledge structure construction}} \\
    & $\circ$ & Drafted a proposal and received funding from The Innovative Research Foundation for Postgraduates, School of Psychology, SCNU \\
    & $\circ$ & Conducted an online experiment and recruited 120+ participants  \\
    & $\circ$ & Performed data processing and statistical analysis using Python and R \\
    
    2020 & \multicolumn{2}{X}{\textbf{Personal topic interest, generative concept sorting, and external structure information}} \\
    & $\circ$ & Wrote an application form for ethical approval \\
    & $\circ$ & Conducted two online experiments and performed statistical analysis using SPSS and JASP \\
    & $\circ$ & Found that concept maps led to better comprehension compared with outlines \\

    2019 & \multicolumn{2}{X}{\textbf{Facial identification in different visual fields}} \\
    & $\circ$ & Coursework of the postgraduate module \textit{Research Method in Psychology}. Lecturer: Prof. Ming Meng \\ 
    & $\circ$ & Arranged the whole study as a team leader \\
    & $\circ$ & Used Matlab to process a hundred and twelve pictures of facial expressions into different spatial frequencies as visual stimuli \\
    & $\circ$ & Designed an academic poster, gave an oral report, and wrote a 3,000+ words course paper, all these works were done in English \\
    

    2018-2019 & \multicolumn{2}{X}{\textbf{The effect of self-depletion on information avoidance in decision making}} \\
    & $\circ$ & BA thesis \\
    % & $\circ$ & Generated the research idea and designed experiments independently by reviewing publications \\
    & $\circ$ & Designed the experiment using E-Prime and performed statistical analysis using SPSS \\
    & $\circ$ & Received a high score in the dissertation defense \\

    \end{tabularx}

%%%%%%%%%%%%%%%%%%%%%%%%%%%%%%%%%%%%%%%%%%%%%%%%%%%%%%%%%%%%%%%
\subsubsection*{ \ \ WORK EXPERIENCE}

\begin{tabularx}{\textwidth}{p{2cm} p{0.1cm} X}

    % 2022 & \multicolumn{2}{X}{\textbf{Research assistant, Tsinghua University}} \\
    % & $ \circ $ & Research asistant in the \href{https://www.ymhlab.com}{youth mental health lab} \\

    2021 & \multicolumn{2}{X}{\textbf{Teaching assistant, South China Normal University}} \\
    & $ \circ $ & Teaching assistant for three undergraduate modules: \textit{Principle and Application of Psychometrics}, \textit{Psychological Statistics}, \textit{The Manipulations and Applications of Statistical Software} \\

    2018 & \multicolumn{2}{X}{\textbf{Practice teacher, Nanchang city}} \\
    & $ \circ $ & Taught mental health lessons to secondary school students

    \end{tabularx}

\  \par 
\  \par 

%%%%%%%%%%%%%%%%%%%%%%%%%%%%%%%%%%%%%%%%%%%%%%%%%%%%%%%%%%%%%%%


\subsubsection*{ \ \ SELECTED HONOURS AND AWARDS}

\begin{tabularx}{\textwidth}{p{2cm} X}

    2021-2022 & Scholarship for postgraduates (Second- and Third-class)\\

    2019 & The Innovative Research Foundation for Postgraduates, School of Psychology, SCNU \\
    
    2019 & Scholarship for undergraduates (Third-class)
    
\end{tabularx}

\  \par

%%%%%%%%%%%%%%%%%%%%%%%%%%%%%%%%%%%%%%%%%%%%%%%%%%%%%%%%%%%%%%%

\subsubsection*{ \ \ PUBLICATIONS}

\begin{description}
    
    \item \textbf{Wei, Z.}, Zhang, Y., Yin, S., Chen, X*., Clariana, R., B*. (manuscript in preparation). Effects of reading purposes on integration from multiple documents: Evidence from concept network analysis.

    \item Chen, X., \textbf{Wei, Z.}, Li, Z., \& Clariana, R. B*. (in revision). Personal topic interest, generative concept sorting, and external structure information. \textit{Educational Technology Research and Development}. [SSCI]

    \item \textbf{Wei, Z.}, Chen, X*., \& Clariana, R., B. (2022). Measures of knowledge structure in reading comprehension. \textit{Journal of Psychological Science}, \textit{45}(2), 306-315. [CSSCI]

    
    \end{description}

\  \par

%%%%%%%%%%%%%%%%%%%%%%%%%%%%%%%%%%%%%%%%%%%%%%%%%%%%%%%%%%%%%%%

\subsubsection*{ \ \ SKILLS}

\begin{tabularx}{\textwidth}{p{2cm} X}
    
    Software & R (proficient), Python (proficient), MATLAB (good), SPSS (good), E-Prime (good), Mplus (basic), Microsoft Words, Excel \& PowerPoint (proficient), LaTeX (good) \\
    
    Language & English (IELTS: 6.5), Chinese (native) \\
    
    \end{tabularx}

\  \par
\  \par

%%%%%%%%%%%%%%%%%%%%%%%%%%%%%%%%%%%%%%%%%%%%%%%%%%%%%%%%%%%%%%%

\subsubsection*{ \ \ REFEREES}

\ \ Assoc. Prof. Xuqian Chen \par
\ \ School of Psychology, South China Normal University \par
\ \ Email: \href{mailto: chenxuqian@m.scnu.edu.cn}{chenxuqian@m.scnu.edu.cn} \par
\ \ Research interest: language cognition and learning; concept structure in memory	\par

\ \par

\ \ Prof. Aitao Lu  \par
\ \ School of Psychology, South China Normal University \par
\ \ Email: \href{mailto: atlupsy@gmail.com}{atlupsy@gmail.com} \par
\ \ Research interest: cognitive process of language; language cognition; teenager development \par


\end{document}